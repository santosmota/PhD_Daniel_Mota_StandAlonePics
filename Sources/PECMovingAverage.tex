%%%%%%%%%%%%%%%%%%%%%%%%%%%%%%%%%%%%%%%%%%%%%%%%%%%%%%%%%%%%%%%%%%%%%%%%%%%%
% Moving average block
% one of the first tikz diagrams I worked with...
%%%%%%%%%%%%%%%%%%%%%%%%%%%%%%%%%%%%%%%%%%%%%%%%%%%%%%%%%%%%%%%%%%%%%%%%%%%%
\tikzstyle{block} = [draw, rectangle, minimum height=1cm, minimum width=1cm]
\tikzstyle{sum} = [draw, circle, node distance=1cm]
\tikzstyle{junction} = [coordinate, node distance=1cm]
\begin{tikzpicture}[auto, node distance=1.5cm,>=latex']
	
    %placing the blocks
	\node [junction, name=input] {};
	\node [block, right of=input, node distance=1.25cm] (gain) {$\frac{1}{N}$};
	\node [junction, right of=gain, node distance=1.5cm] (junta1) {};
	\node [junction, right of=junta1] (topoZn) {};
	\node [block, below of=topoZn] (zN) {$z^{-N}$};
	\node [junction, below of=zN] (fromPLL) {};
	\node [sum, right of=topoZn] (sum1) {$\Sigma$};
	\node [junction, right of=sum1, node distance=2.4cm] (meio) {};
	\node [sum, right of=meio] (sum2) {$\Sigma$};
	\node [junction, right of=sum2] (topoZ1) {};
	\node [above of=topoZ1, node distance=0.75cm] (acctitle) {Accumulator};
	\node [block, below of=topoZ1] (z1) {$z^{-1}$};
	\node [junction, right of=topoZ1] (junta2) {};
	\node [junction, right of=junta2] (output) {};

	\coordinate [above of=meio, node distance=1cm] (topdotted);
	\coordinate [below of=meio, node distance=2cm] (bottdotted);
	\draw [dotted] (topdotted) -- (bottdotted);
	
	\draw [->] (input) -- node[pos=0.0] {$x(k)$} (gain);
	\draw [->] (gain) -- node[pos=0.4] {$x_{\textrm{acc}}=\frac{x(k)}{N}$} node[pos=1, anchor=south east] {$+$} (sum1);
	\draw [->] (junta1) |- (zN);
	\draw [->] (zN) -| node[near end, anchor=west] {$x_{\textrm{dec}}=\frac{x(k-N)}{N}$} node[pos=1, anchor=north east] {$-$} (sum1);
	\draw [->] (sum1) -- node [pos=0.35] {$x_{\textrm{acc}_{\textrm{in}}}(k)$} node[pos=1, anchor=south east] {$+$} (sum2);
	\draw [->] (z1) -| node[pos=1, anchor=north east] {$+$} (sum2);
	\draw [->] (junta2) |- (z1);
	\draw [->] (sum2) -- node[pos=1] {$\overline{x}(k)$} (output);

    \fill (junta1) circle [radius=1pt];
    \fill (junta2) circle [radius=1pt];
 
\end{tikzpicture}
